%  spec.tex
%  Document created by seblovett on seblovett-NETBOOK
%  Date created: Fri 18 Apr 2014 19:50:22 BST
%  <+Last Edited: Sat 19 Apr 2014 12:22:16 BST by seblovett on seblovett-Ubuntu +>

\documentclass[12pt]{article}

\usepackage[nodayofweek]{datetime}
\usepackage{listings}
\author{Henry Lovett, Sam Hooper}

\title{Car Track Specification}

\begin{document}
\maketitle

%\begin{abstract}
%Abstract
%\end{abstract}

\section{Introduction}

This project is about creating a module capable of remotely tracking a car. 
The module is capable of acquiring and reporting a location to a website. 

\section{Specification}
The following are the main goals:

\begin{enumerate}
\item Be able to report the location of the car to a website. 
\item Website shows past and current location of the car.
\item Hardware device uses GPS to acquire location and GSM to send data to a website.
\item Hardware must be able to send data irrespective of whether the car is running.
\item Hardware is capable of being configured by a user via the website.
\end{enumerate}

\section{Hardware Device}

The hardware is to be a module with a GPS and GSM modem. 
It is to be powered by the car battery. 
LiPo battery could be used, but the current draw should be low enough to not impact the main battery. 
It will be concealed in the car.

\subsection{Specification}

\begin{enumerate}
\item Device should be able to acquire location to within an acceptable accuracy
\item Device can send data over an internet GSM connection
\item Module should have a distance threshold. Once the threshold is passed, the new location is reported.
\item During sleep mode, GSM should be disabled to save battery. Website should be polled to check for option updates. GPS should be polled to check the car hasn't moved.
\item Device should report to website when low on credit (if PAYG SIM card is used)
\end{enumerate}


\subsection{Costings}
The below is an approximate costings for one module. 

\begin{table}[!h]
\begin{tabular}{ccc}
Item		&	Quantity	& 	Cost (\pounds) \\ \hline
Qualtec M10 GSM & 	1		& 	15.00 \\
MTK3339 GPS	&	1		&	10.00 \\
Atmega 644P QFP & 	1		&	6.00  \\
PCB (10cm by 10cm)&	10		&	20.00 \\
Other passives	&	N/A		&	5.00 \\ \hline
TOTAL		&	1		&	56 \\ \hline
TOTAL 		&	2		&	87 \\ \hline
\end{tabular}
\end{table}


\section{Website}

Website should be able to show the locations for a car. 
Log in and user should be supported. 
A module should be able to be configured with options.
A ``My Car Is Stolen'' button could be implemented to heighten the sleep poll rate.
Be able to send an alert if the credit is low. 

\section{Interface}

``Sending'' data is the moving of information from the module to the server.
``Receiving'' data is the moving of information from the server to the module.

\subsection{Sending Data}

A webpage will be used to send data to via a query string. 
The query string should be of the form:

\begin{lstlisting}
www.url.co.uk/interface.php?uid=01&lat=0.01&long=-0.01&credit=1.01
\end{lstlisting}

The interface.php webpage should report ``OK'' if the query string is well formed and the UID is recognised. 
Else, ``ER'' should be returned. 
The text should be plain text and not HTML formatted.
Credit reporting is optional. 

\subsection{Receiving Data}

A webpage will be used to configure the device.
The page will be accessed with a query string of the device as follows:

\begin{lstlisting}
www.url.co.uk/options.php?uid=01
\end{lstlisting}

The page should look up the configuration from a database and return plain text in the following format:

\begin{lstlisting}
OPTION=1
\end{lstlisting}
For example:

\begin{lstlisting}
WAKE_PERIOD=2
SLEEP_PERIOD=60
MOVE_THRESH=100
\end{lstlisting}

Exact option names and units of the values are TBC. 


\end{document}

